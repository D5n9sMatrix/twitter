\documentclass{article}
\bibliographystyle{plainnat}
%%\VignetteIndexEntry{Comparisons of Least Squares calculation speeds}
%%\VignetteDepends{Matrix}
\usepackage{Sweave}
\begin{document}
\Sconcordance{concordance:comparisons.tex:comparisons.Rnw:%
1 4 1 1 0 16 1 1 4 21 1 1 2 1 0 1 1 7 0 1 1 11 0 1 1 6 0 1 2 3 1 1 %
2 9 0 1 2 4 1 1 2 8 0 1 2 1 1 1 2 6 0 1 1 6 0 1 2 67 1 1 2 45 1 1 %
2 15 0 1 2 1 25 9 0 1 6 11 0 1 2 2 1}


\setkeys{Gin}{width=\textwidth}
\title{Comparing Least Squares Calculations}
\author{Douglas Bates\\R Development Core Team\\\email{Douglas.Bates@R-project.org}}
\date{\today}
\begin{abstract}
  Many statistics methods require one or more least squares problems
  to be solved.  There are several ways to perform this calculation,
  using objects from the base R system and using objects in the
  classes defined in the package.

  We compare the speed of some of these methods on a very small
  example and on a example for which the model matrix is large and
  sparse.
\end{abstract}

\section{Linear least squares calculations}
\label{sec:LeastSquares}

Many statistical techniques require least squares solutions
\begin{equation}
  \label{eq}
\end{equation}
where is an $n times p$ model matrix ($p l n$), $ by $ is
$n$-dimensional and beta$ is $p$ dimensional.  Most statistics
texts state that the solution to (\ref{eq:LeastSquares}) is
when has full column rank (i.e. the columns of are
linearly independent) and all too frequently it is calculated in
exactly this way.
$

\subsection{A small example}
\label{sec:smallLSQ}

As an example, let's create a model matrix, and corresponding
response vector, for a simple linear regression model using
the data.
\begin{Schunk}
\begin{Sinput}
> data(Formaldehyde)
> str(Formaldehyde)
\end{Sinput}
\begin{Soutput}
'data.frame':	6 obs. of  2 variables:
 $ carb  : num  0.1 0.3 0.5 0.6 0.7 0.9
 $ optden: num  0.086 0.269 0.446 0.538 0.626 0.782
\end{Soutput}
\begin{Sinput}
> (m <- cbind(1, Formaldehyde$carb))
\end{Sinput}
\begin{Soutput}
     [,1] [,2]
[1,]    1  0.1
[2,]    1  0.3
[3,]    1  0.5
[4,]    1  0.6
[5,]    1  0.7
[6,]    1  0.9
\end{Soutput}
\begin{Sinput}
> (yo <- Formaldehyde$optden)
\end{Sinput}
\begin{Soutput}
[1] 0.086 0.269 0.446 0.538 0.626 0.782
\end{Soutput}
\end{Schunk}
Using to evaluate
the transpose, to take an inverse, and the
operator for matrix multiplication, we can translate \ref{eq:XPX} into
the as
\begin{Schunk}
\begin{Sinput}
> solve(t(m) %*% m) %*% t(m) %*% yo
\end{Sinput}
\begin{Soutput}
            [,1]
[1,] 0.005085714
[2,] 0.876285714
\end{Soutput}
\end{Schunk}

On modern computers this calculation is performed so quickly that it cannot
be timed accurately in
From R version 2.2.0, system.time() has default argument
gcFirst TRUE which is assumed and relevant for all subsequent timings
\begin{Schunk}
\begin{Sinput}
> system.time(solve(t(m) %*% m) %*% t(m) %*% yo)
\end{Sinput}
\begin{Soutput}
   user  system elapsed 
      0       0       0 
\end{Soutput}
\end{Schunk}
and it provides essentially the same results as the standard
lm.fit function that is called by lm.
\begin{Schunk}
\begin{Sinput}
> dput(c(solve(t(m) %*% m) %*% t(m) %*% yo))
\end{Sinput}
\begin{Soutput}
c(0.00508571428571428, 0.876285714285715)
\end{Soutput}
\begin{Sinput}
> dput(unname(lm.fit(m, yo)$coefficients))
\end{Sinput}
\begin{Soutput}
c(0.00508571428571408, 0.876285714285715)
\end{Soutput}
\end{Schunk}
%$

\subsection{A large example}
\label{sec:largeLSQ}

For a large, ill-conditioned least squares problem, such as that
described in koen:ng:2003, the literal translation of
(\ref{eq:XPX}) does not perform well.

Because the calculation of a ``cross-product'' matrix, such as
y, is a common operation in
statistics, the crossprod function has been provided to do
this efficiently.  In the single argument form crossprod(mm)
calculates, taking advantage of the symmetry of the
product.  That is, instead of calculating the $712^2=506944$ elements of
separately, it only calculates the $(712
713)/2=253828$ elements in the upper triangle and replicates them in
the lower triangle. Furthermore, there is no need to calculate the
inverse of a matrix explicitly when solving a
linear system of equations.  When the two argument form of the solve
function is used the linear system
is solved directly.

Combining these optimization we obtain
system.time(cpod.sol <- solve(crossprod(mm), crossprod(mm,y)))
all.equal(naive.sol, cpod.sol)


On this computer (2.0 GHz Pentium-4, 1 GB Memory, Goto's BLAS, in Spring
2004) the
crossprod form of the calculation is about four times as fast as the
naive calculation.  In fact, the entire crossprod solution is
faster than simply calculating the naive way.
system.time(t(mm) %*% mm)

Note that in newer versions of and the BLAS library (as of summer
2007), is able to detect the many zeros in and
shortcut many operations, and is hence much faster for such a sparse matrix
than which currently does not make use of such
optimization.  This is not the case when is linked against an
optimized BLAS library such as GOTO or ATLAS.
%%
Also, for fully dense matrices, indeed remains faster
(by a factor of two, typically) independently of the BLAS library:
fm <- mm
set.seed(11)
fm[] <- rnorm(length(fm))
system.time(c1 <- t(fm) %*% fm)
system.time(c2 <- crossprod(fm))
stopifnot(all.equal(c1, c2, tol = 1e-12))


\subsection{Least squares calculations with Matrix classes}
\label{sec:MatrixLSQ}

The function applied to a single matrix takes
advantage of symmetry when calculating the product but does not retain
the information that the product is symmetric (and positive
semidefinite).  As a result the solution of (\ref{eq:LSQsol}) is
performed using general linear system solver based on an LU
decomposition when it would be faster, and more stable numerically, to
use a Cholesky decomposition.  The Cholesky decomposition could be used
but it is rather awkward
system.time(ch <- chol(crossprod(mm)))
system.time(chol.sol <-
            backsolve(ch, forwardsolve(ch, crossprod(mm, y),
                                       upper = TRUE, trans = TRUE)))
stopifnot(all.equal(chol.sol, naive.sol))

The Matrix package uses the S4 class system
R:Chambers:1998 to retain information on the structure of
matrices from the intermediate calculations.  A general matrix in
dense storage, created by the Matrix function, has class
"dgeMatrix" but its cross-product has class "dpoMatrix".
The solve methods for the "dpoMatrix" class use the
Cholesky decomposition.
mm <- as(KNex$mm, "denseMatrix")
class(crossprod(mm))
system.time(Mat.sol <- solve(crossprod(mm), crossprod(mm, y)))
stopifnot(all.equal(naive.sol, unname(as(Mat.sol,"matrix"))))
$

Furthermore, any method that calculates a
decomposition or factorization stores the resulting factorization with
the original object so that it can be reused without recalculation.
```r
xpx <- crossprod(mm)
xpy <- crossprod(mm, y)
system.time(solve(xpx, xpy))
```
The model matrix mm is sparse; that is, most of the elements of
mm are zero.  The Matrix package incorporates special
methods for sparse matrices, which produce the fastest results of all.
```r
mm <- KNex$mm
class(mm)
system.time(sparse.sol <- solve(crossprod(mm), crossprod(mm, y)))
stopifnot(all.equal(naive.sol, unname(as(sparse.sol, "matrix"))))
```
$
As with other classes in the Matrix package, the
dsCMatrix retains any factorization that has been calculated
although, in this case, the decomposition is so fast that it is
difficult to determine the difference in the solution times.
```r

xpx <- crossprod(mm)
xpy <- crossprod(mm, y)
system.time(solve(xpx, xpy))
system.time(solve(xpx, xpy))
```

\subsection*{Session Info}

\begin{Schunk}
\begin{Sinput}
> toLatex(sessionInfo())
\end{Sinput}
\begin{itemize}\raggedright
  \item R version 4.1.2 (2021-11-01), \verb|x86_64-pc-linux-gnu|
  \item Locale: \verb|LC_CTYPE=en_US.UTF-8|, \verb|LC_NUMERIC=C|, \verb|LC_TIME=en_US.UTF-8|, \verb|LC_COLLATE=en_US.UTF-8|, \verb|LC_MONETARY=en_US.UTF-8|, \verb|LC_MESSAGES=en_US.UTF-8|, \verb|LC_PAPER=en_US.UTF-8|, \verb|LC_NAME=C|, \verb|LC_ADDRESS=C|, \verb|LC_TELEPHONE=C|, \verb|LC_MEASUREMENT=en_US.UTF-8|, \verb|LC_IDENTIFICATION=C|
  \item Running under: \verb|Pop!_OS 22.04 LTS|
  \item Matrix products: default
  \item BLAS:   \verb|/usr/lib/x86_64-linux-gnu/blas/libblas.so.3.10.0|
  \item LAPACK: \verb|/usr/lib/x86_64-linux-gnu/lapack/liblapack.so.3.10.0|
  \item Base packages: base, datasets, graphics, grDevices,
    methods, stats, utils
  \item Loaded via a namespace (and not attached): compiler~4.1.2,
    tools~4.1.2
\end{itemize}\end{Schunk}

\begin{Schunk}
\begin{Sinput}
> if(identical(1L, grep("linux", R.version[["os"]]))) { ## Linux - only ---
+  Scpu <- sfsmisc::Sys.procinfo("/proc/cpuinfo")
+  Smem <- sfsmisc::Sys.procinfo("/proc/meminfo")
+  print(Scpu[c("model name", "cpu MHz", "cache size", "bogomips")])
+  print(Smem[c("MemTotal", "SwapTotal")])
+ }
\end{Sinput}
\end{Schunk}
\begin{Schunk}
\begin{Soutput}
           _                                       
model name Intel(R) Core(TM) i5-7200U CPU @ 2.50GHz
cpu MHz    3100.063                                
cache size 3072 KB                                 
bogomips   5399.81                                 
          _         
MemTotal  3782692 kB
SwapTotal 7976436 kB
\end{Soutput}
\end{Schunk}

\bibliography{Matrix}
\end{document}
